\documentclass{article}

\usepackage[margin=1.5in]{geometry} % Please keep the margins at 1.5 so that there is space for grader comments.
\usepackage{amsmath,amsthm,amssymb,hyperref}

\newcommand{\R}{\mathbf{R}}  
\newcommand{\Z}{\mathbf{Z}}
\newcommand{\N}{\mathbf{N}}
\newcommand{\Q}{\mathbf{Q}}

\newenvironment{theorem}[2][Theorem]{\begin{trivlist}
\item[\hskip \labelsep {\bfseries #1}\hskip \labelsep {\bfseries #2.}]}{\end{trivlist}}
\newenvironment{lemma}[2][Lemma]{\begin{trivlist}
\item[\hskip \labelsep {\bfseries #1}\hskip \labelsep {\bfseries #2.}]}{\end{trivlist}}
\newenvironment{claim}[2][Claim]{\begin{trivlist}
\item[\hskip \labelsep {\bfseries #1}\hskip \labelsep {\bfseries #2.}]}{\end{trivlist}}
\newenvironment{problem}[2][Problem]{\begin{trivlist}
\item[\hskip \labelsep {\bfseries #1}\hskip \labelsep {\bfseries #2.}]}{\end{trivlist}}
\newenvironment{proposition}[2][Proposition]{\begin{trivlist}
\item[\hskip \labelsep {\bfseries #1}\hskip \labelsep {\bfseries #2.}]}{\end{trivlist}}
\newenvironment{corollary}[2][Corollary]{\begin{trivlist}
\item[\hskip \labelsep {\bfseries #1}\hskip \labelsep {\bfseries #2.}]}{\end{trivlist}}

\newenvironment{solution}{\begin{proof}[Solution]}{\end{proof}}

\begin{document}

\large % please keep the text at this size for ease of reading.

% ------------------------------------------ %
%                 START HERE             %
% ------------------------------------------ %

{\Large Joey Cheung
\hfill  Special Lecture Notes}

% -----------------------------------------------------
% The "enumerate" environment allows for automatic problem numbering.
% To make the number for the next problem, type " \item ". 
% To make sub-problems such as (a), (b), etc., use an "enumerate" within an "enumerate."
% -----------------------------------------------------

\begin{enumerate}

\item Partition

\begin{enumerate}
    \item $Max-Cut \in{NP}$
    \item For a graph, a maximum cut is a cut whose size is at least the size of any other cut. The problem of finding a maximum cut in a graph is known as the Max-Cut Problem. 
    \item $Min-Cut \in{P}$
    \item In graph theory, a minimum cut or min-cut of a graph is a cut that is minimal in some sense
    \item 2-partition. Karp uses 2 partition but the standard is using K partitions (Base 10 to base K)
    \item $\pi$ {is another way of saying partition}
    \item partition $\pi(S)$
    \item k-partition $\pi(S,K)$
    \item The partition problem, or number partitioning, is the task of deciding whether a a multiset of positive integers can be partitioned into two subsets.
    \item $US_i = S$
\end{enumerate}

\item Graphs
\begin{enumerate}
    \item $G = (V,E)$
    \item $E \subset{VxV}$
\end{enumerate}

\item Formula for partition in Rosen's book

$$(n^m \sum_{i=1}^{n-1} (-1) ^{i+1} {n \choose i} (n - i)^m)/n!$$

\item Another formula

$$2^n  = \sum_{i=n}^n {n \choose i} $$

\item Permutation 
\begin{enumerate}
    \item Permutation has i! permutations for 1 combination
    \item $ n!/(n-i)!$
\end{enumerate}

\item Combination
\begin{enumerate}
    \item Combination formula
    \item $n!/(n-i)! n$
\end{enumerate}

\item Mutually Exclusive
\begin{enumerate}
    \item Intersectional requirement, may not span "U"
\end{enumerate}

\item Problems
\begin{enumerate}
    \item Set packing: L boxes - no duplication
    \item Set packing is a classical NP-complete problem in computational complexity theory and combinatorics, and was one of Karp's 21 NP-complete problems.
    \item Set covering: Boxes – covers all items “U”, (But may not be mutually exclusive)
    \item The set cover problem is a classical problem in computer science and is one of Karp's 21 NP-complete problems
    \item Exact cover: Up to Hboxes
    \item The exact cover problem is NP-complete and is one of Karp's 21 NP-complete problems. The exact cover problem is a kind of constraint satisfaction problem. The exact problem is a problem that sees if you can have an exact cover.
\end{enumerate}

\end{enumerate}

\end{document}